\documentclass{article}

\usepackage{listings}
\usepackage{color}

\title{Notes on C/C++ Programming}
\author{Dan Ibanez}
\begin{document}

\definecolor{mygreen}{rgb}{0,0.6,0}
\lstdefinestyle{myc}{
basicstyle=\footnotesize\ttfamily,
keywordstyle=\bfseries\color{mygreen}
}
\lstset{language=C++,style=myc}

\maketitle

\section{The Interface}

In many cases, we would like to be able to have multiple providers of a
predefined set of services.
For example, multiple geometric modelers or mesh databases.
This pattern is also useful for plugging user-defined code into a
general-purpose system.
Callbacks are, for all intensive purposes, the same thing as an interface.

An interface is a collection of functions set up in such a way that code
can be written that calls those functions, and the implementations of the
functions can be changed without modifying the code.

Technically regular functions fit this description and, as long as the overall
behavior is not changed, the implementation can be replaced by recompiling
the code and linking with a different library.
MPI is an example of this setup.
Programs that call MPI functions can be linked agains MPICH or OpenMPI
without modification.
However, the ``one provider at a time" limit is often too restrictive.
It works well when the services provided are functional rather than
data-oriented.
MPI doesn't store much, it only passes messages.
On the other hand, if we would like to convert a geometric model from
one representation to another and both implementations have the same
function names, it is very difficult to create a code that calls them
both at once.

For this reason, interfaces should ideally be dynamic, which means there
is some form of lookup or dispatch at runtime depending on which provider
is being used.
We will begin by looking at C++ virtual functions, and then implement
a very similar system in C.

Lets look at a simple interface which stores a string.

\begin{lstlisting}
struct StringStore
{
  virtual ~StringStore() {}
  virtual void store(const char* s) = 0;
  virtual const char* retrieve() = 0;
};

struct MyStore : public StringStore
{
  MyStore() {str = NULL;}
  ~MyStore() {free(str);}
  void store(const char* s) {str = strdup(s);}
  const char* retrieve() {return str;}
  const char* str;
};
\end{lstlisting}

We use \verb+struct+ instead of \verb+class+ because it saves us having
to denote things as \verb+public+.
The \verb+StringStore+ class is the interface.
It defines what the \verb+store+ and \verb+retrieve+ functions look like.
The virtual destructor needs to be there so that the following situation
will properly free the string:

\begin{lstlisting}
StringStore* ss = new MyStore();
ss->store("hello");
delete ss;
\end{lstlisting}

The last line calls \verb+~StringStore+, but because it is virtual
this correctly dispatches to \verb+~MyStore+, which frees the string.

The \verb+= 0+ after each \verb+StringStore+ function means there is no
default implementation, which makes them pure virtual, which makes
\verb+StringStore+ an abstract base class.
This means you can't make a plain old \verb+StringStore+ object, since
it would have no code in its functions.

Now lets look at how the C++ compiler implements all this \verb+virtual+
business, and in the process derive a more flexible C implementation.
First, C++ class methods have access to the object on which they act
through a hidden argument called \verb+this+.
For example, the function \verb+MyStore::store+ eventually ends up
looking something like this:

\begin{lstlisting}
void MyStore_store(MyStore* this, const char* s)
{
  this->str = strdup(s);
}
\end{lstlisting}

Virtual function dispatch is typically implemented via a virtual
function table, or \verb+vtable+ for short.
The base class, \verb+StringStore+, defines what this table
should look like:

\begin{lstlisting}
struct StringStore_vtable
{
  void (*destroy)(struct StringStore* this);
  void (*store)(struct StringStore* this, const char* s);
  const char* (*retrieve)(struct StringStore* this);
};
\end{lstlisting}

It is a structure containing pointers to functions.
The functions all have the implicit \verb+this+ argument.
Also, the \verb+destroy+ function represents the virtual destructor.

Each object derived from \verb+StringStore+ must then carry a pointer
to this \verb+vtable+ around.
Programmers should be aware that objects which have virtual functions
start with a hidden pointer, which will affect design decisions in terms
of memory use.

\begin{lstlisting}
struct StringStore
{
  struct StringStore_vtable* vtable;
};

struct MyStore
{
  struct StringStore base;
  const char* str;
};
\end{lstlisting}

When the C++ code calls \verb+ss->store+, it is translated to
something like the following:

\begin{lstlisting}
ss->vtable.store(ss, "hello");
\end{lstlisting}

In the case of \verb+MyStore+, its implementation would look something like
this:

\begin{lstlisting}
void MyStore_destroy(struct StringStore* this)
{
  free( ((MyStore*)this)->str );
}

void MyStore_store(struct StringStore* this, const char* s)
{
  ((MyStore*)this)->str = s;
}

const char* MyStore_retrieve(struct StringStore* this)
{
  return ((MyStore*)this)->str;
}

struct StringStore_vtable MyStore_vtable =
{
  .destroy = MyStore_destroy,
  .store = MyStore_store,
  .retrieve = MyStore_retrieve
};

StringStore* MyStore_create(void)
{
  struct MyStore* p = malloc(sizeof(struct MyStore));
  p->base.vtable = &MyStore_vtable;
  p->str = NULL;
  return &p->base;
}
\end{lstlisting}

Clearly this is a bit more verbose, which is why the C++ version is preferable
for users with less programming experience.
However, the C++ syntax is just a shorthand which the compiler translates
into this \verb+vtable+ and function system.

With the dive into a lower level, naturally, comes greater flexibility.
Notice that now we have the opportunity to fill in the interface through
this \verb+vtable+ structure.
This is more powerful than the inheritance tree system offered by C++.

Assume we have a mostly functional interface, and we would like to mix and
match the best functions from many different providers:

\begin{lstlisting}
struct animal_skills chimera_skills =
{
  .run = cheetah_run,
  .swim = dolphin_swim,
  .fly = eagle_fly
};
\end{lstlisting}

Doing this with C++ inheritance would likely require a more complicated
setup that obscures the original intention.
However, trying to picture what this animal looks like gives a sense of
the data structure issues that mixing and matching can cause, if the
interface is more data-oriented than functional.

In addition to free-form mixing, we can also omit functions with relative
ease: just set their pointer to zero.
This is useful in large interfaces where not all providers can provide
all services.
Checking for services is a simple pointer comparison.
Although C++ can do something similar with empty base class implementations,
it does not allow true omission, since that would be an abstract base class,
and so there is no mechanism in C++ to check whether the function exists,
it just has to.

The C version of this interface system is used throughout the Linux kernel.
A notable example of this use is file systems.
Each file system provides services like creating, renaming, and deleting
files and directories.
Each of these services then becomes a function pointer in a structure,
implemented differently by each file system.
In addition, advanced features like asynchronous data transfer are
also function pointers, which some file systems set to zero if they
cannot provide them.

\section{The Function Table}

In the last section we covered \verb+vtable+ structures, which are actually
more structure than table.
In this case, we look at a similar pattern which is a proper table of
function pointers with the same type.
There are situations when we must pick one of many numbered choices.
A switch statement can solve this problem, but if each choice contains
lots of code then it usually results in a giant block of code.
If the cases are separated out into functions, this starts to look much
like the table we are about to present anyway.

Consider this enumeration of element types in a finite element simulation:

\begin{lstlisting}
enum {
  VERTEX,
  EDGE,
  TRIANGLE,
  QUADRILATERAL,
  TETRAHEDRON,
  HEXAHEDRON
};

int getType(Entity* e);
\end{lstlisting}

During mesh adaptation, we would like to choose how to split an entity
based on its type.
We can do this using a type table like this:

\begin{lstlisting}
void split(Entity* e)
{
  int type = getType(e);
  typedef void (*SplitFunction)(Entity* e);
  static SplitFunction table[6] =
  {splitVertex,
   splitEdge,
   splitTriangle,
   splitQuadrilateral,
   splitTetrahedron,
   splitHexahedron};
  table[type](e);
}
\end{lstlisting}

C++ users will immediately protest: why not use a virtual function ?
The Entity object can have a virtual function called split, whose
implementation can be defined by the sub-classes such as Triangle.
Assuming Entity and Triangle already exist like this, that is a reasonable
objection for a small code with a few such functions to dispatch.
Now consider the existence of one or two dozen such operations.
Making them all virtual functions starts to clutter the class definitions.
In addition, it requires the class definition to know about everything
everyone will ever do to said class, which is not very extensible.
Finally, there are cases where the decision is based on something
other than ``object type", loosely defined, such as a number describing
one of many possible states of each element.

\section{Workers}

Another special case of the Interface pattern comes up when there
is considerable boilerplate code involved in what is essentially iteration,
and programmers would prefer not to copy and paste that boilerplate
every time something different is done for each item.
More generally, there exist complex algorithms, slight variations of which
produce very different but equally useful results.
Classic examples include tree and graph traversals, but those may even be
too simple to really justify this method.

The setup begins with identifying the bits of the code that do not change
and writing them in a general way once and for all.
The parts that do change, usually the code that is executed when an item
is visited, are put into an Interface.

The interface for a graph traversal might look as follows:

\begin{lstlisting}
struct GraphTraversal
{
  virtual bool visited(Node* n);
  virtual void visit(Node* n);
};

void breadthFirstSearch(Graph* g, GraphTraversal* t);
void depthFirstSearch(Graph* g, GraphTraversal* t);
\end{lstlisting}

The mesh adaptation code has a Worker called a Crawler, which acts on
every quadrilateral in the boundary layer of the mesh, one layer at a
time, correctly handling boundary layers that are separated by partitioning.
The iteration code lives in one place, and the Crawler interface provides
user-defined code to invoke at each entity and across part boundaries.

\section{Class Hiding}

In C++, if a user wants access to methods of a class, they must include
a file that also describes the internal variables of a class.
This in turns requires including descriptions of the types of those
variables, which may be classes that in turn include other classes,
and so on.

In C, a structure can be separated from the functions that act on it
by declaring it only as an opaque pointer.

\begin{lstlisting}
struct genie;
struct genie* make_genie(void);
void make_wish(struct genie* g);
void free_genie(struct genie* g);
\end{lstlisting}

The same thing can be done in C++, in which case it is sometimes called
``forward declaration" of a class.
To reduce the amount of internals the user has to deal with, it may be
beneficial to write free functions acting on opaque classes, even if they
are just wrappers over class methods.

\begin{lstlisting}
class Genie;
Genie* makeGenie();
void makeWish(Genie* g);
void freeGenie(Genie* g);
\end{lstlisting}

\section{Fake Pointers}

Recall that Interfaces can be used to make two very different implementations
of the same services look exactly the same on the outside.
If these interfaces involve handling many small objects or entities,
one runs into the case where these objects have very different types in
each implementation.
They may be objects of unrelated type or in some implementations they may
not be objects at all and instead are referenced by array indices.

If they were all objects then there is a chance of using a C++ inheritance
system, but that falls apart when the implementations are developed by someone
else who is in all likelihood not interested in deriving from your wrapper
classes.

In the case of integer identifiers, assuming that the range $[0,n-1]$ is
valid and the null identifier is -1, we can convert the integers to pointers
and back by adding or subtracting one.
This has the benefit that the null identifier translates to zero as a pointer,
which is the {\it de facto} standard value for null pointers.

\begin{lstlisting}
namespace all {
class Sound;
struct Listener
{
  virtual Sound* listen();
};
}

namespace one {
class Noise;
Noise* getNextNoise();
struct Listener : public all::Listener
{
  virtual Sound* listen() {return (Sound*)getNextNoise();}
};
}

int two_make_phonon();

namespace two {
struct  Listener : public all::Listener
{
  virtual Sound* listen()
  {
    int n = two_make_phonon();
    return (Sound*)(((char*)0) + (n + 1));
  }
};
\end{lstlisting}

\section{Dependency Inversion}

A truly great use for interfaces is to decouple high and low level components
by inverting their dependencies.
Traditionally, we would think that high-level code would depend on low-level
components which it uses directly.
This works fine until there arise multiple competing implementations for
a low-level component, or even when the single third-party provider changes
their functions significantly.

A wise alternative is to isolate each low-level component behind an Interface.
That Interface is not the native one provided by the low-level component,
but rather a common agreement between the low-level and high-level component.
In fact, the low-level component rarely has to agree or care, so long as it
provides the requisite services in some form.

A high-level component should identify exactly what it expects from the lower
levels and create its ideal Interface for those services.
Then, that Interface is fulfilled using the native functions of each low-level
component.

The first immediate benefit is that the high-level component no longer depends
on any low-level component, only on the Interface.
The low-level component can be built separately from and later than
the high-level code.
The Interface implementation for a particular low-level component can also
be compiled separately, after both the high-level and low-level components
are compiled.
This can be done for multiple providers of the same service, or even none at
all.

If a native function of or the dominant provider of a low-level service
changes, then the Interface protects all the high-level code from that change.

Interfaces are analogous to standards like Internet communication
protocols, building material sizes, file formats, and national currencies.
Standards enable interoperability, which in turn makes the whole system
much more efficient.

Note that there is no need for a single governing standard.
Like national currencies, there may be multiple standards, each one reaching
as far as the people who agree to use it.
Connecting two different interfaces is fairly trivial, like exchange rates
between national currencies, as opposed to asking Russia to change their whole
economy to use Chinese currency, for example.

By defining an Interface, a group makes a formal and clear statement about
its expectations for a particular low-level component, and immediately
protects all their code against future changes in that component.
The best Interface will be adaptable to all providers which satisfy
the overall service needs.

Provider lock-in is a very serious danger for many codes today.
Realize that calling the native functions of another code directly is a
very committed relationship, and will force the caller to bear with
unnecessary hardship from the provider due to the increasingly high
cost and risk of the surgical operation to replace all such calls.

A low-level interface is something you choose for life.
You may as well define it yourself as a set of ideal qualities,
not as any specific provider at the time.

One last theoretical note: this process changes a strict high-to-low
dependency into something more powerful.
The dependency graph can now have cycles, because all components (graph nodes)
are built first and then the interfaces (graph edges) between them are built.
Gone are the headaches associated with trying to resolve cyclic dependencies
between components, since everything is decoupled.

\section{Arrays}

Alright, not quite a design pattern.
However, great programmers including Rob Pike have made it clear that
programming is very much guided by the data structures used, so it is well
worth discussing them.
Most programs need containers, which are schemes for storing and accessing
many objects.
A great variety of data structures qualify as containers, and many
can become quite intricate and special-purpose.

The argument made here is that is that structures should be built from
arrays if at all possible.
Professor Chuck Steward once said that when using the C++ STL, one should
try to use \verb+std::vector+ first, then \verb+std::list+, then
\verb+std::map+.
These are implementations of growable arrays, linked lists, and red-black
trees, respectively.

If the array's size is known at compile time, it can be declared directly
without using memory allocation functions.
This is useful for manipulating small sets and saves time with functions
that take arrays as arguments.

\begin{lstlisting}
void sort(int* array, int size);
int main()
{
  int numbers[5] = {0,2,1,3,4};
  sort(numbers, 5);
}
\end{lstlisting}

If the array's size is given at runtime and does not change during its
lifetime, it can be created and destroyed using a variety of memory
allocation functions (\verb+malloc/free+, \verb+new/delete+, ...).

Finally, if the array is expected to change in size, there are several
tricks that can be employed.

The first is geometric growth.
Suppose we use an array to implement a stack.
Users push entities onto the stack and remove them from the top.
If the user pushes enough entities to fill the current array, we
can allocate a new and bigger array, copy all items from the old
to the new array, then deallocate the old array.
If the new array is only one item larger, this expensive process
will repeat itself every time the user pushes an item.
Instead, we can choose a constant factor greater than one and multiply
the current array size by this factor to get the new array size.
This ensures that we have room left over for several more pushes
before the expensive process kicks in.
If the growth uses multiplication, one can prove that the cost
of pushing $n$ items takes $O(n)$ time, as opposed to $O(n^2)$
when growing only by one.
C++'s \verb+std::vector+ uses a growth factor of 2.
The Git version control program uses the formula $n' = \frac32(n + 16)$.
Notice that there is balance to strike since extra room
improves performance for possible later pushes but wastes space
in the present.

The second is the use of C's \verb+realloc+ function.
This function behaves as though it makes a new array, copies content,
and frees the old one, but leaves open the possibility of a
more efficient implementation.
On systems with virtual memory for instance, using the POSIX
\verb+mmap+ system to handle large arrays allows \verb+realloc+
to grow an array without copying any items, simply by changing
the way physical RAM maps to virtual memory.

\end{document}
